\documentclass[12pt,a4paper]{scrartcl} 
\usepackage[utf8]{inputenc}
\usepackage[english,russian]{babel}
\usepackage{indentfirst}
\usepackage{misccorr}
\usepackage{graphicx}
\usepackage{indentfirst}
\usepackage{amsmath}
\begin{document}
 \begin{titlepage}
  \begin{center}
   \large
   МИНИСТЕРСТВО НАУКИ И ВЫСШЕГО ОБРАЗОВАНИЯ РОССИЙСКОЙ ФЕДЕРАЦИИ
   
   Федеральное государственное бюджетное образовательное учреждение высшего образования
   
   \textbf{АДЫГЕЙСКИЙ ГОСУДАРСТВЕННЫЙ УНИВЕРСИТЕТ}
   \vspace{0.25cm}
   
   Инженерно-физический факультет
   
   Кафедра автоматизированных систем обработки информации и управления
   \vfill

   \vfill
   
   \textsc{Отчет по практике}\\[5mm]
   
   {\LARGE Программаная реализация определителя матрицы. \textit{Вариант 7}}
   \bigskip
   
   1 курс, группа 1ИВТ1-1
  \end{center}
  \vfill
  
  \newlength{\ML}
  \settowidth{\ML}{«\underline{\hspace{0.7cm}}» \underline{\hspace{2cm}}}
  \hfill\begin{minipage}{0.5\textwidth}
   Выполнил:\\
   \underline{\hspace{\ML}} Д.\,Д.~Давтян\\
   «\underline{\hspace{0.7cm}}» \underline{\hspace{2cm}} 2023 г.
  \end{minipage}%
  \bigskip
  
  \hfill\begin{minipage}{0.5\textwidth}
   Руководитель:\\
   \underline{\hspace{\ML}} С.\,В.~Теплоухов\\
   «\underline{\hspace{0.7cm}}» \underline{\hspace{2cm}} 2023 г.
  \end{minipage}%
  \vfill
  
  \begin{center}
   Майкоп, 2023 г.
  \end{center}
 \end{titlepage}
 
% Содержание
\section{Введение}
\label{sec:intro}


\subsection{Формулировка цели}
Целью данной работы является написание программы для нахождение определителя матрицы.

\subsubsection{Теория}
Определителем квадратной матрицы A = \begin{pmatrix}
a_{11}&a_{12} \\
a_{21}&a_{22}
\end{pmatrix} второго порядка называется число |A|= {}a_{11}a_{22} - a_{12}a_{21}.

Определителем A=\begin{pmatrix}
a_{12}& \cdots & a_{1n} \\
\vdots & \ddots & \vdots \\
a_{n1} & \cdots & a_{nn}
\end{pmatrix}
квадратной матрицы порядка $n,n \geq 3$, называется число $|A|= \sum_{}^{}^{n}_{k=1}(-1)^{k+1}a_{1k}M_{k}$, где $M_{k} - $ определитель матрицы порядка n$ - $1, полученной из матрицы A вычеркиванием первой строки и столбца с номером k.

\section{Ход работы}
\subsection{Код выполненной программы}
\sloppy
\begin{verbatim}

def calculate_determinant(matrix):
    n = len(matrix)

    # Базовый случай: матрица 1x1
    if n == 1:
        return matrix[0][0]

    determinant = 0

    for column in range(n):
        submatrix = [[matrix[i][j] for j in range(n) \
                      if j != column] for i in range(1, n)]
        coefficient = (-1) ** column
        determinant += coefficient * matrix[0][column] \
                       * calculate_determinant(submatrix)

    return determinant

# Ввод размерности матрицы
r = int(input("Введите размерность квадратной матрицы: "))

# Ввод элементов матрицы
m = []
for i in range(r):
    row = []
    for j in range(r):
        element = int(input(f"Введите элемент [{i+1},{j+1}]: "))
        row.append(element)
    m.append(row)

# Вычисление определителя
determinant = calculate_determinant(m)

print("Определитель матрицы:", determinant)


\end{verbatim}

\begin{figure}[h]
 \centering
 \includegraphics[width=0.8\textwidth]{result.png}
 \caption{Результат работы}\label{fig:par}
\end{figure}

\end{document}